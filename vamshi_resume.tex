                                                                     
                                                                     
                                                                     
                                             
%%%%%%%%%%%%%%%%%%%%%%%%%%%%%%%%%%%%%%%%%%%%%%%%%%%%%%%%%%%%%%%%%%%%%%%%
%%%%%%%%%%%%%%%%%%%%%% Simple LaTeX CV Template %%%%%%%%%%%%%%%%%%%%%%%%
%%%%%%%%%%%%%%%%%%%%%%%%%%%%%%%%%%%%%%%%%%%%%%%%%%%%%%%%%%%%%%%%%%%%%%%%

%%%%%%%%%%%%%%%%%%%%%%%%%%%%%%%%%%%%%%%%%%%%%%%%%%%%%%%%%%%%%%%%%%%%%%%%
%% NOTE: If you find that it says                                     %%
%%                                                                    %%
%%                           1 of ??                                  %%
%%                                                                    %%
%% at the bottom of your first page, this means that the AUX file     %%
%% was not available when you ran LaTeX on this source. Simply RERUN  %%
%% LaTeX to get the ``??'' replaced with the number of the last page  %%
%% of the document. The AUX file will be generated on the first run   %%
%% of LaTeX and used on the second run to fill in all of the          %%
%% references.                                                        %%
%%%%%%%%%%%%%%%%%%%%%%%%%%%%%%%%%%%%%%%%%%%%%%%%%%%%%%%%%%%%%%%%%%%%%%%%

%%%%%%%%%%%%%%%%%%%%%%%%%%%% Document Setup %%%%%%%%%%%%%%%%%%%%%%%%%%%

% Don't like 10pt? Try 11pt or 12pt
\documentclass[11pt]{article}     
\marginparwidth = -5pt
% This is a helpful package that puts math inside length specifications
\usepackage{calc}
%\usepackage[T1]{fontenc}
\usepackage{graphicx}

% Simpler bibsection for CV sections
% (thanks to natbib for inspiration)
\makeatletter
\newlength{\bibhang}
\setlength{\bibhang}{1em}
\newlength{\bibsep}
 {\@listi \global\bibsep\itemsep \global\advance\bibsep by\parsep}
\newenvironment{bibsection}%
        {\begin{list}{}{%
       \setlength{\leftmargin}{\bibhang}%
       \setlength{\itemindent}{-\leftmargin}%
       \setlength{\itemsep}{\bibsep}%
       \setlength{\parsep}{\z@}%
        \setlength{\partopsep}{0pt}%
        \setlength{\topsep}{0pt}}}
        {\end{list}\vspace{-.6\baselineskip}}
\makeatother

% Layout: Puts the section titles on left side of page
\reversemarginpar

%
%         PAPER SIZE, PAGE NUMBER, AND DOCUMENT LAYOUT NOTES:
%
% The next \usepackage line changes the layout for CV style section
% headings as marginal notes. It also sets up the paper size as either
% letter or A4. By default, letter was used. If A4 paper is desired,
% comment out the letterpaper lines and uncomment the a4paper lines.
%
% As you can see, the margin widths and section title widths can be
% easily adjusted.
%
% ALSO: Notice that the includefoot option can be commented OUT in order
% to put the PAGE NUMBER *IN* the bottom margin. This will make the
% effective text area larger.
%
% IF YOU WISH TO REMOVE THE ``of LASTPAGE'' next to each page number,
% see the note about the +LP and -LP lines below. Comment out the +LP
% and uncomment the -LP.
%
% IF YOU WISH TO REMOVE PAGE NUMBERS, be sure that the includefoot line
% is uncommented and ALSO uncomment the \pagestyle{empty} a few lines
% below.
%

%% Use these lines for letter-sized paper
\usepackage[%paper=letterpaper,
    %        includefoot, % Uncomment to put page number above margin
            marginparwidth=1.1in,     % Length of section titles
            marginparsep=.07in,       % Space between titles and text
            %margin=0.5in,               % 1 inch margins
            top=1in, bottom=1.5in, left=0.4in, right=0.5in,
             includemp]{geometry}
	

%% Use these lines for A4-sized paper
%\usepackage[paper=a4paper,
%            %includefoot, % Uncomment to put page number above margin
%            marginparwidth=30.5mm,    % Length of section titles
%            marginparsep=1.5mm,       % Space between titles and text
%            margin=25mm,              % 25mm margins
%            includemp]{geometry}




%% Use these lines for A4-sized paper
%\usepackage[paper=a4paper,
%            %includefoot, % Uncomment to put page number above margin
%            marginparwidth=30.5mm,    % Length of section titles
%            marginparsep=1.5mm,       % Space between titles and text
%            margin=25mm,              % 25mm margins
%            includemp]{geometry}

%% More layout: Get rid of indenting throughout entire document
\setlength{\parindent}{0in}
\textheight = 11in
\usepackage[shortlabels]{enumitem}
%\usepackage[hidelinks]{hyperref}
%% Reference the last page in the page number
%
% NOTE: comment the +LP line and uncomment the -LP line to have page
%       numbers without the ``of ##'' last page reference
%
% NOTE: uncomment the \pagestyle{empty} line to get ridof all page
%       numbers (make sure includefoot is commented out above)
%
\voffset = -0.29in
\usepackage{fancyhdr,lastpage}
\pagestyle{fancy}
\pagestyle{empty}      % Uncomment this to get rid of page numbers
\fancyhf{}\renewcommand{\headrulewidth}{0pt}
\fancyfootoffset{\marginparsep+\marginparwidth}
\newlength{\footpageshift}
\setlength{\footpageshift}
          {0.5\textwidth+0.5\marginparsep+0.5\marginparwidth-2in}
\lfoot{\hspace{\footpageshift}%
       \parbox{4in}{\, \hfill %
                    \arabic{page} of \protect\pageref*{LastPage} % +LP
%                   \arabic{page}                               % -LP
                    \hfill \,}}

% Finally, give us PDF bookmarks
\usepackage{color,hyperref}
\definecolor{darkblue}{rgb}{0.0,0.0,0.3}
\hypersetup{breaklinks,%colorlinks=false,hidelinks,
            linkcolor=darkblue,urlcolor=darkblue,
            anchorcolor=darkblue,citecolor=darkblue}

\usepackage{multirow}
%%%%%%%%%%%%%%%%%%%%%%%% End Document Setup %%%%%%%%%%%%%%%%%%%%%%%%%%%%


%%%%%%%%%%%%%%%%%%%%%%%%%%% Helper Commands %%%%%%%%%%%%%%%%%%%%%%%%%%%%

% The title (name) with a horizontal rule under it
% (optional argument typesets an object right-justiied across from name
%  as well)
%
% Usage: \makeheading{name}
%        OR
%        \makeheading[right_object]{name}
%
% Place at top of document. It should be the first thing.
% If ``right_object'' is provided in the square-braced optional
% argument, it will be right justified on the same line as ``name'' at
% the top of the CV. For example:
%
%       \makeheading[\emph{Curriculum vitae}]{Your Name}
%
% will put an emphasized ``Curriculum vitae'' at the top of the document
% as a title. Likewise, a picture could be included:
%
%   \makeheading[\includegraphics[height=1.5in]{my_picutre}]{Your Name}
%
% the picture will be flush right across from the name.
\newcommand{\makeheading}[2][]%
        {\hspace*{-\marginparsep minus \marginparwidth}%
         \begin{minipage}[t]{\textwidth+\marginparwidth+\marginparsep}%
             {\large \bfseries #2 \hfill #1}\\[-0.15\baselineskip]%
                 \rule{\columnwidth}{1pt}%
         \end{minipage}}

% The section headings
%
% Usage: \section{section name}
\renewcommand{\section}[1]{\pagebreak[3]%
    \hyphenpenalty=10000%
    \vspace{1.3\baselineskip}%
    \phantomsection\addcontentsline{toc}{section}{#1}%
    \noindent\llap{\scshape\smash{\parbox[t]{\marginparwidth}{\raggedright #1}}}%
    \vspace{-\baselineskip}\par}

% An itemize-style list with lots of space between items
\newenvironment{outerlist}[1][\enskip\textbullet]%
        {\begin{itemize}[#1,leftmargin=*]}{\end{itemize}%
         \vspace{-.6\baselineskip}}

% An environment IDENTICAL to outerlist that has better pre-list spacing
% when used as the first thing in a \section
\newenvironment{lonelist}[1][\enskip\textbullet]%
        {\begin{list}{#1}{%
        \setlength{\partopsep}{0pt}%
        \setlength{\topsep}{0pt}}}
        {\end{list}\vspace{-.6\baselineskip}}

% An itemize-style list with little space between items
\newenvironment{innerlist}[1][\enskip\textbullet]%
        {\begin{itemize}[#1,leftmargin=*,parsep=0pt,itemsep=0pt,topsep=0pt,partopsep=0pt]}
        {\end{itemize}}

% An environment IDENTICAL to innerlist that has better pre-list spacing
% when used as the first thing in a \section
\newenvironment{loneinnerlist}[1][\enskip\textbullet]%
        {\begin{itemize}[#1,leftmargin=*,parsep=0pt,itemsep=0pt,topsep=0pt,partopsep=0pt]}
        {\end{itemize}\vspace{-.6\baselineskip}}

% To add some paragraph space between lines.
% This also tells LaTeX to preferably break a page on one of these gaps
% if there is a needed pagebreak nearby.
\newcommand{\blankline}{\quad\pagebreak[3]}
\newcommand{\halfblankline}{\quad\vspace{-0.5\baselineskip}\pagebreak[3]}

% Uses hyperref to link DOI
\newcommand\doilink[1]{\href{http://dx.doi.org/#1}{#1}}
\newcommand\doi[1]{doi:\doilink{#1}}

% For \url{SOME_URL}, links SOME_URL to the url SOME_URL
\providecommand*\url[1]{\href{#1}{#1}}
% Same as above, but pretty-prints SOME_URL in teletype fixed-width font
\renewcommand*\url[1]{\href{#1}{\texttt{#1}}}

% For \email{ADDRESS}, links ADDRESS to the url mailto:ADDRESS
\providecommand*\email[1]{\href{mailto:#1}{#1}}
% Same as above, but pretty-prints ADDRESS in teletype fixed-width font
%\renewcommand*\email[1]{\href{mailto:#1}{\texttt{#1}}}

%\providecommand\BibTeX{{\rm B\kern-.05em{\sc i\kern-.025em b}\kern-.08em
%    T\kern-.1667em\lower.7ex\hbox{E}\kern-.125emX}}
%\providecommand\BibTeX{{\rm B\kern-.05em{\sc i\kern-.025em b}\kern-.08em
%    \TeX}}
\providecommand\BibTeX{{B\kern-.05em{\sc i\kern-.025em b}\kern-.08em
    \TeX}}
\providecommand\Matlab{\textsc{Matlab}}

%%%%%%%%%%%%%%%%%%%%%%%% End Helper Commands %%%%%%%%%%%%%%%%%%%%%%%%%%%

%%%%%%%%%%%%%%%%%%%%%%%%% Begin CV Document %%%%%%%%%%%%%%%%%%%%%%%%%%%%

\begin{document}
\makeheading{\Large \sc KURVA VAMSHI KUMAR}
%\makeheading { {ABID K. \\ india \\ indj fda} \hfill \fbox{\includegraphics[scale=0.13] {newface.jpg}} }
%\\MS-Applicant,EE, Fall 2013 (U-M ID 93342488)

%        {\hspace{-\marginparsep minus \marginparwidth}
%         \begin{minipage}{\textwidth+\marginparwidth+\marginparsep}
         
%         \begin{tabular}{l r  }
         
%\huge \bfseries Abid K \\

%  & &  \\
%Karumannil(H), Iringalloor P.O. & \hspace{0.5in}  \\
%Malappuram, Kerala &  \textit{E-mail:} \email{abidrahman2@gmail.com}\\
% India - 676304 & \textit {Mobile:} +91-9740660498, +91-9496370018
%\end{tabular}

 %       \rule{\columnwidth}{1pt}
%		\end{minipage} 




\section{Personal Information}

% NOTE: Mind where the & separators and \\ breaks are in the following
%       table.
%
% ALSO: \rcollength is the width of the right column of the table
%       (adjust it to your liking; default is 1.85in).
%
\newlength{\rcollength}\setlength{\rcollength}{3.2in}%
\begin{tabular}[t]{@{}p{\textwidth-\rcollength}p{\rcollength}}
%\href{http://www.cse.osu.edu/}%
{Yalal, Tandur} & \textit{Mobile:} +91-9550535220\\
%%\href{http://www.osu.edu/}
{Rangareddy, Telangana , India - 501144} & \textit{E-mail:} \email{vamshikumarkurva@gmail.com}\\ 
% & \textit{Date of Birth:} 17 February 1988\\

%& \textit{Website:} \href{URL}{URL}

\end{tabular}

\vspace{0.1in}

%\section{Objective}
%Placement in an academic position that allows for research and development and imparting my knowledge and skills to the students.
%Admission to a top graduate school for doctoral studies.
% with a particular focus on modelling, design, and fabrication.

\vspace{-0.1in}

\section{Research Interests}
Computer Vision, Machine Learning, Pattern recognition
%Computer Vision, Machine Learning, Computer Architecture, Digital Systems Design


%\vspace{-4mm}
\section{Education}\itemsep -5pt
        
%\vspace{0.05in}        

\href{www.iist.ac.in}{\textbf{Indian Institute of Space Science and Technology, Valiyamala}}\hfill \textbf{Pursuing sem-II}\\
{\it M.Tech, Machine Learning and Soft Computing} 
        \begin{innerlist}
        \item \sl CGPA: \bf\emph{9.0}
%        \item \sl ELECTIVES: \emph{Analog and Digital VLSI Design,Digital Signal Processing,Introduction to MEMS,Power Electronics.}\vspace{0.05in}
        \end{innerlist}
        
\href{www.sreenidhi.edu.in}{\textbf{Sreenidhi  Institute of Science and Technology, Ghatkesar}}\hfill \textbf{2010 - 2014}\\
{\it B.Tech, Electronics and Communication Engineering} 
        \begin{innerlist}
        \item \sl Percentage: \bf\emph{84.14}
%        \item \sl ELECTIVES: \emph{Analog and Digital VLSI Design,Digital Signal Processing,Introduction to MEMS,Power Electronics.}\vspace{0.05in}
        \end{innerlist}
        
        \vspace{0.05in}        
        
{\textbf{Sri Chaitanya Junior College,                   
Ameenpur}}\hfill \textbf{2008 - 2010}\\
{\it Intermediate, MPC} 
        \begin{innerlist}
        \item \sl Percentage: \bf\emph{96.5}
        \end{innerlist}
        
        \vspace{0.05in}
        
        {\textbf{A.P.Residential School,
Keesaragutta}}\hfill \textbf{2008}\\
{\it SSC} 
        \begin{innerlist}
        \item \sl Percentage: \bf\emph{93}
        \end{innerlist}



\vspace{-0.1in}

\section{Industrial Experiance}
\textbf{Design Engineer, Auviz Systems}  \hfill {\bf September 2014 - July 2015} \\
\textbf{Job Description} : FPGA based implementation of Computer Vision algorithms.\\
\textbf{Projects} : Object Tracking, FFT, Fixed Point Coding\\
\textbf{Keywords} : FPGA, Computer Vision, Machine Learning

\vspace{-0.1in}

\section{Relevant Courses}
\textbf{Electronics:} \textup{Digital Electronics, Control Systems, Communication systems }\\
\textbf{Machine Learning:} \textup{ANN, Datamining, Kernel Methods}\\
\textbf{Mathematics:} \textup{Probability Theory and Random Processes, Optimization techniques,Descrete Mathematics and Differential Equations}

\vspace{-0.1in}

\section{Projects Undertaken}

\textbf{Image Compression using 2DPCA}\\
M.Tech mini Project
\begin{innerlist}
%\item Main goal is the automated detection of characteristic points of ECG.
\item The aim of this project is to extract the features from an image removing the redundant information by measuring the correlation between rows and columns of the image using PCA.
\item Tools: OpenCV, python
\end{innerlist}

\vspace{.1in}

\textbf{Object Tracking}\\
Industry Project
\begin{innerlist}
\item Given an object in the first frame of video, track it in the subsequent frames using the Mean Shift Algorithm.
\item Tools: OpenCV, vivado HLS
\end{innerlist}

\section{Technical Skills}
\textbf{Programming Languages:} \textup{C, C++, python}\\
\textbf{Software Packages:} \textup{\Matlab/Octave, HLS, OpenCV }

\textbf{VLSI Tools:} \textup{Xilinx Vivado HLS, Proteus, Tina }

\textbf{Operating Systems:} \textup{Linux(CentOS, Ubuntu), Windows}

\vspace{-0.1in}

\section{Talks}
\textbf{``Singular Value Decomposition(SVD)''}, M.Tech Seminar Talk, IIST \hfill \textbf{ March 2016}

\vspace{-0.1in}

\section{Activities} 
\begin{innerlist}
\item Participated in  \textbf{Hardware Training-Electronic System Design} conducted by \textbf{INSIGNIA LABS}.
\item Co-ordinator for \textbf{ADASTRA 2012}, Tech-Fest conducted by SNIST.
\end{innerlist}

\section{References}
Available upon request

%\section{References}
%\begin{tabular}{ l l l }
%  \textbf{Dr. Deepu Vijayasenan} & \hspace{1in} & \textbf{Alexander Mordvintsev}  \\
%  Assistant Professor & \hfill & Senior Software Engineer  \\
%  ECE Department & \hfill & GIS Department, Transas \\
%  NITK, Suratkal, India & \hfill & St. Petersburg, Russia \\
%  \textit{Email}: \href{deepu.senan@gmail.com}{deepu.senan@gmail.com} & \hfill & \textit{Email}: \href{Alexander.Mordvintsev@transas.com}{Alexander.Mordvintsev@transas.com} \\
%\end{tabular}

\end{document}


