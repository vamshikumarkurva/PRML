\documentclass[12pt,a4paper]{article}
\title{Pattern Recognition Machine Learning}
\author{vamshi, SC15M045}
\usepackage{amsmath}
\usepackage{amssymb}
\usepackage{graphicx}
\usepackage{float}
\usepackage{flafter}
\usepackage{geometry}
\usepackage{lmodern}
\usepackage[utf8]{inputenc}
\usepackage[T1]{fontenc}
\begin{document}
	\pagenumbering{gobble}
	\maketitle
	\newpage
	\section{}
	\subsection*{(a)}
	An n dimensional Euclidean space with norm defined as $ \|x\|_p = {(\sum {{|x_i|}^p})}^{\dfrac{1}{p}}  $ is a normed space but not inner product space except for p=2, because this norm doesn't satisfy the \textbf{Parallelogram Equality} required of a norm to have an inner producy associated with it. \newline
	\textbf{Parallelogram Equality:} \newline
	In normed space, the statement of parallelogram law is an equation relating norms\newline
	\hspace*{10ex} $ \|x+y\|^2 + \|x-y\|^2 = 2\|x\|^2 + 2\|y\|^2 $ \newline
	For any norm satisfying the parallelogram law, the inner product generating the norm is unique and is given by \newline
	\hspace*{10ex} $ <x,y> = \dfrac{\|x+y\|^2 - \|x-y\|^2}{4} $
	\subsection*{(b)}
	 $ f,g \in C[0,1], f(x) = x+1 , g(x) = sin(x) $\newline
	 $ <f,g> = \int_{a}^{b} f(x).g(x) dx \newline \hspace*{8ex} =\int_{0}^{1} (x+1).sin(x) dx \newline \hspace*{8ex} = [-(x+1)cos(x) + sin(x)]_{0}^{1}$
	 
	 \section{}
	 M be a closed subspace of Hilbert space H(complete inner product space). $ M \subset H $ \newline
	 let $ \{X_n\} $ be the cauchy sequence in M $ \implies $ $ \{X_n\} $ be the cauchy sequence in H. Since H is complete  $ {X_n} \longrightarrow x \in H $. x is the limit point of M.\newline  since M is closed, $ x \in M $. i.e, Every cauchy sequence in M, converges in M.\newline $\therefore $  M is complete $ \implies $ M is Hilbert space.
	 
	 \section{}
	 Training points $(x_i, y_i), i=1,2,...N $. Let $ f \in F(RKHS) $ generates the data.f is entirely contained in $span\{k_{x_i}\}$. let Y = $span\{k_{x_i}\}$. As every finite dimensional subspace of a normed space
	 X is closed in X, Y is closed.Therefore by projection theorem,
	 \begin{center}
	   $	F = Y \oplus Y^\perp $ 
	 \end{center}
	 Hence $ f = f_y + f_{y^\perp}, f_y \in Y, f_{y^\perp} \in Y^{\perp} $. Now $f(x_i) = ({<f, K_{x_i}>}) = (<f_y, k_{x_i})>$. As $ f_y \in Y, f_y = \sum_{i=1}^{i=n} \alpha_i k_{x_i}$. Thus any function of the form $f_y = \sum_{i=1}^{i=n} \alpha_i k_{x_i} + f', f' \in Y^{\perp}$ satisfies the given points.
	 
	 \section{}
	 \subsection*{(a)}
	 $f(x) = 3x_1+4x_2+5x_3, x=(x_1,x_2,x_3)$
	 $f(x) = w^T x, w = w = (3,4,5)^T; $. 
	 \begin{center}
	 	$\|f\| = \|w\| = 5\sqrt{2}$
	 \end{center}
	 
	 \subsection*{(b)}
	 Let $M = \{k_{x_i}\}$ and let $ f \in M^{\perp} $. Therefore $ <f,k_x> = 0, \forall x \in X$. Therefore f(x) = 0 $\forall x \in X$. Hence f $\equiv 0 $.Hence $ M^{\perp} = \{0\} $. Hence $ \overline{span(M)}=F $. Therefore every $ f \in F $ can be represented as 
	 \begin{center}
	 	$f = \sum \alpha_i k_{x_i}, \alpha_i \in \mathbb{R}$ 
	 \end{center} 
	 Hence $f(x) = \sum \alpha_i k_{x_i}(x) = \sum \alpha_i k(x_i,x) $. Therefore every element in RKHS can be represented as linear combination of $k_{x_i}$
	 
	 \subsection*{(c)}
	 \subsubsection*{(i)}
	 $\tilde{f}(x) = f(x) + b, f \in \mathbb{F}, $
	 Reproducing kernel $ k(x,y) = {<x,y>}^2$ 
	 \begin{equation*}
	 \begin{split}
	 	 	k(x,y) & = {(x_1y_1+x_2y_2)}^2 \\
	 	 	& = \begin{bmatrix}
	 	 	    x_1^2 \\ x_2^2 \\ \sqrt{2}x_1x_2
	 	 	    \end{bmatrix}.\begin{bmatrix}
	 	     	y_1^2 \\ y_2^2 \\ \sqrt{2}y_1y_2
	 	    	\end{bmatrix}
	 	      = \hspace*{2ex}<\phi(x), \phi(y)>
	 \end{split}
	 \end{equation*}
	 $k(x,x) = 100, k(x,y)=k(y,x)=9, k(y,y)=1$.
	 any $ f \in \mathbb{F} $ can be represented as $ f = \sum_{i=1}^{i=1} k_{x_i}$. Therefore\newline
	 
	 $ \|f\|^2  = (<f,f>) 
	            = (<\alpha_1k_{x_1}+\alpha_2k_{x_2}, \alpha_1k_{x_1}+\alpha_2k_{x_2} >) $\newline
	 \hspace*{9ex}= $\alpha_1^2k(x,x)+2\alpha_1\alpha_2k(x,y)+\alpha_2^2k(y,y)$\newline
	 \hspace*{9ex}=$100\alpha_1^2+18\alpha_1\alpha_2+\alpha_2^2$\newline
	 unknown parameters are $\alpha_1, \alpha_2, b$
	 \subsubsection*{(ii)}
	 $\tilde{f}$ is in the linear span of $k_{x_i}$ only if b=0, but f is in the linear span of $k_{x_i}$.\newline
	 Kernel methods finds a unique f and unique b such that,it minimizes the regularized risk function and  f has the minimum norm of all functions that can be solutions. 
	          
\end{document}